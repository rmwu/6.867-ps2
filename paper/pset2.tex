%\documentclass{amsart}

\documentclass{article}
\usepackage[letterpaper,hmargin=15mm,vmargin=20mm]{geometry}
\usepackage[nosetup, colorlinks]{tony}
\usepackage{graphicx}

\usepackage{amsmath,amssymb}
\usepackage{mathpazo}
\usepackage{multicol}
\usepackage{diagbox}

\usepackage{xcolor}
%\usepackage[printwatermark]{xwatermark}
%\newwatermark*[allpages,color=gray!50,angle=45,scale=3,xpos=0,ypos=0]{DRAFT}

\DeclareMathOperator{\sgn}{sgn}
\newcommand{\sind}[1]{^{(#1)}}

\title{6.867: Problem Set 2}
\date{October 25, 2016}

\begin{document}
\maketitle

\begin{multicols}{2}

% % % % % % % % % %
%    PROBLEM 1
% % % % % % % % % %

\section{Logistic Regression}



% % % % % % % % % %
%    PROBLEM 2
% % % % % % % % % %

\section{Support Vector Machines (SVM)}

We implemented the dual form of soft-SVM
with third-party convex optimization software.
Recall that the soft-SVM dual optimization problem takes the form
\begin{equation}
    \label{eq:soft-svm-dual}
    \begin{array}{ll@{}ll}
        \text{maximize}  &\displaystyle -\f{1}{2}\sum_{i,j} \alpha_i \alpha_j y\sind{i} y\sind{j} \angb{x\sind{i}, x\sind{j}}
        +
        \sum_i \alpha_i \\
        \text{subject to}& 0 \le \alpha_i \le C\\
        & \displaystyle\sum_{i=1}^n \alpha_i y\sind{i} = 0
    \end{array}
\end{equation}
for training data $x\sind{i}, y\sind{i}$ ($i=1, \dots, n$),
with labels $y\sind i = \pm 1$.
Given a new data point $x$,
we would predict its label $y$ as
\begin{equation}
    \label{eq:soft-svm-predict}
    y = \sgn\lt(\sum_i \alpha^*_i y\sind{i} \angb{x\sind{i}, x}\rt)
\end{equation}
where $\alpha^*_i$ is the optimal value of $\alpha_i$
from the previous optimization.

% TODO
% Show in your report the constraints and objective that you generate for the 2D problem with positive examples (2, 2), (2, 3) and negative examples (0, -1), (-3, -2). Which examples are support vectors?

We tested our soft-SVM implementation
on the same four 2D datasets from the previous section.
% TODO
% makes sure 2D datasets are mentioned in previous section
Setting the regularization parameter $C=1$,
we get % TODO fill in


\subsection{Kernelization}

It is often useful to map our raw data
with some nonlinear feature map $\phi$
into a higher dimensional feature space.
However, it is often infeasible to compute and store in memory
all the values $\phi(x\sind{i})$.

Therefore, we recall the ``kernel trick": the insight that
because Equation~\ref{eq:soft-svm-dual} and \ref{eq:soft-svm-predict}
only refer to values $\phi(x\sind{i})$ inside inner products
with other $\phi(x\sind{j})$,
we only need the kernel function
\begin{equation}
    k(x, x') = \angb{\phi\lt(x\rt), \phi\lt(x'\rt)},
\end{equation}
which is usually much easier to compute than the feature map $\phi$.

By replacing the inner products in
Equations~\ref{eq:soft-svm-dual} and \ref{eq:soft-svm-predict}
with the corresponding kernels,
we arrive at the kernelized soft-SVM optimization problem.

We tested our kernelized SVM routine with a linear kernel
\begin{equation}
    k(x, x') = 1 + x\cdot x'
\end{equation}
and with Gaussian RBF kernels
\begin{equation}
    k(x, x') = \exp\lt(-\f{|x - x'|^2}{2\sigma^2}\rt)
\end{equation}
with varying bandwidth~$\sigma$.  % TODO how did sigma vary?
We chose values of $C$ in
\[
    C \in \{0.01, 0.1, 1, 10, 100\}
\]

% TODO
% Report your results and answer the following questions:
%
%(a) What happens to the geometric margin 1/|w| as C increases? Will this always happen as we increase C?
%(b) What happens to the number of support vectors as C increases?
%
%(c) The value of C will typically change the resulting classifier and therefore also affects the accuracy on test examples. Why would maximizing the geometric margin 1/||w|| on the training set not be an appropriate criterion for selecting C? Is there an alternative criterion that we could use for this purpose?



% % % % % % % % % %
%    PROBLEM 3
% % % % % % % % % %

\section{SVMs with Pegasos}

% % % % % % % % % %
%    PROBLEM 4
% % % % % % % % % %

\section{MNIST Digit Recognition}


\end{multicols}

\end{document}
